\chapter{Background}

This is what the thesis will discuss.

\section{Autonomous Agents}
\textit{Autonomous Agents: A Revolutionary Paradigm in Artificial Intelligence }

In recent years, the field of Artificial Intelligence (AI) has witnessed a paradigmatic
shift with the emergence of autonomous agents. These intelligent systems have
the ability to perceive their environment, reason about it, and take actions to
achieve their goals without the need for explicit human supervision or
intervention. Autonomous agents are designed to operate in complex, dynamic
environments, where they must adapt and learn from experience to navigate and succeed.

The concept of autonomy in AI is closely related to the idea of self-awareness ,
where an agent possesses some level of consciousness or awareness of its own internal
state and external environment. This self-awareness enables the agent to monitor
its progress, identify areas for improvement, and adjust its behavior
accordingly. In the context of autonomous agents, autonomy refers to the ability
of the agent to operate independently, making decisions based on its own reasoning
and decision-making processes.

Autonomous agents can be broadly classified into two categories: \textit{semi-autonomous}
and \textit{fully-autonomous}. Semi-autonomous agents are designed to operate within
predetermined parameters or constraints, whereas fully-autonomous agents have
the ability to learn and adapt without any pre-defined guidelines or rules. Fully-autonomous
agents are considered more sophisticated and capable than their semi-autonomous counterparts,
as they can navigate complex environments and make decisions based on their own
reasoning.

The development of autonomous agents has far-reaching implications for various
fields, including robotics, natural language processing, and game playing. For
instance, autonomous robots can be used to perform tasks such as surveillance,
search and rescue operations, and environmental monitoring. In the realm of
natural language processing, autonomous agents can be used to build conversational
AI systems that can understand and respond to human queries. Similarly, in the field
of game playing, autonomous agents can be used to develop more sophisticated AI
opponents.

Some of the key characteristics of autonomous agents include:

\begin{itemize}
  \item \textit{\textbf{Autonomy}}: The ability of an agent to operate independently
    without explicit human supervision or intervention.

  \item \textit{\textbf{Self-awareness}}: The ability of an agent to possess some
    level of consciousness or awareness of its own internal state and external environment.

  \item \textit{\textbf{Reasoning and decision-making}}: The ability of an agent
    to make decisions based on its own reasoning and decision-making processes.

  \item \textit{\textbf{Adaptability}}: The ability of an agent to adapt to changing
    environments and learn from experience.
\end{itemize}

In the context of this project, autonomous agents will play a crucial role in
the development of novel architectures for intelligent systems. By exploring the
capabilities and limitations of autonomous agents, we can develop more
sophisticated AI systems that are capable of operating in complex, dynamic environments.

To further understand the concept of autonomy in AI, it is essential to delve
into the technical aspects of autonomous agent design and development. This includes
the use of machine learning algorithms, natural language processing techniques, and
other forms of artificial intelligence.

\section{BDI Architecture}
The Belief-Desire-Intention (BDI) architecture is a popular framework for developing
autonomous agents that can reason about their own mental states and take appropriate
actions in complex environments. The BDI architecture was first introduced by
Anand Rao and Michael Georgeff in the 1990s, and it has since been widely used in
various applications, including robotics, artificial intelligence, and human-computer
interaction.

Belief

The first component of the BDI architecture is Belief (B). This represents the
agent's mental state, which includes all the information that the agent knows or
believes to be true about its environment. In other words, belief is the agent's
internal representation of reality. The beliefs are typically represented as a
set of propositions, where each proposition is a statement that can be either true
or false.

For example, if we have an autonomous robot that needs to navigate through a
room, its beliefs might include:

The location of the robot (e.g., "I am in the living room") The presence of obstacles
(e.g., "There is a chair in front of me") The goal of the task (e.g., "My goal is
to reach the kitchen") The Belief component is responsible for managing and updating
the agent's mental state based on new information or observations. This involves
reasoning about the meaning of the new information, combining it with existing beliefs,
and revising the belief set accordingly.

Desire

The second component of the BDI architecture is Desire (D). This represents the
agent's goals, preferences, and motivations. Desires are high-level representations
of what the agent wants to achieve or attain in a given situation. Unlike
beliefs, desires are not necessarily based on factual information but rather on the
agent's subjective preferences and values.

For example, if we have an autonomous robot that needs to clean up a room, its desires
might include:

Cleaning up the room completely (e.g., "I want to clean up every single piece of
trash") Avoiding certain areas or objects (e.g., "I don't want to go near the breakable
vase") The Desire component is responsible for evaluating the agent's goals and preferences
in light of its current situation. This involves reasoning about what actions
are likely to achieve the desired outcome, taking into account any constraints or
limitations.

Intention

The third component of the BDI architecture is Intention (I). This represents the
agent's commitment to take a specific course of action towards achieving its goals.
An intention is a plan of action that is explicitly stated and committed to by
the agent. Unlike beliefs, which are merely held true or false, intentions are explicit
commitments to take a particular action.

For example, if we have an autonomous robot that needs to clean up a room, its intention
might be:

"I intend to clean up every single piece of trash in this room." The Intention
component is responsible for translating the agent's desires into concrete
actions. This involves reasoning about what steps are necessary to achieve the desired
outcome, taking into account any constraints or limitations.

Key Features and Properties

Some key features and properties of the BDI architecture include:

Autonomy : The BDI architecture allows agents to operate independently, making
decisions based on their own mental states rather than relying on external
inputs. Reasoning : The BDI architecture enables agents to reason about their own
mental states, including beliefs, desires, and intentions. Flexibility : The BDI
architecture allows agents to adapt to changing situations and environments by revising
their beliefs, desires, and intentions as needed. Completeness : The BDI
architecture ensures that the agent's actions are consistent with its mental
state, avoiding inconsistencies or contradictions. Applications

The BDI architecture has been applied in various domains, including:

Autonomous robotics: The BDI architecture is used to develop autonomous robots that
can navigate complex environments and perform tasks based on their own mental states.
Artificial intelligence: The BDI architecture is used to develop AI systems that
can reason about their own mental states and take appropriate actions. Human-computer
interaction: The BDI architecture is used to develop interfaces that allow humans
to interact with autonomous agents, taking into account the agent's mental state.
Comparison with Other Architectures

The BDI architecture has been compared with other popular architectures, including:

BPE (Basic Processing Element) : The BDI architecture is similar to the BPE in its
use of propositional logic for representing beliefs and desires. However, the BDI
architecture adds an intention component that allows agents to make explicit commitments.
Reinforcement Learning : The BDI architecture can be seen as a form of
reinforcement learning, where the agent learns through trial and error and updates
its mental state accordingly.

\section{Large Language Models}
In this section, we will provide a technical explanation of Large Language
Models, including their attention mechanism and log probabilities.

Overview of LLMs Large Language Models (LLMs) are a type of artificial intelligence
model that is specifically designed to process and understand human language.
These models use complex algorithms and large datasets to learn the patterns and
structures of language, allowing them to generate coherent and contextually relevant
text.

The primary goal of an LLM is to predict the next word in a sequence, given the
context provided by the previous words. This is achieved through a process called
masked language modeling, where some of the words in a sentence are randomly
removed or "masked," and the model must predict the missing words.

Attention Mechanism The attention mechanism is a key component of LLMs that allows
them to focus on specific parts of the input sequence when generating output.
This is particularly useful for tasks such as machine translation, where the model
needs to pay attention to certain words or phrases in the source language to produce
accurate translations.

In essence, the attention mechanism allows the model to weigh the importance of
different input elements (such as words or tokens) based on their relevance to
the current task. This is achieved through a process called self-attention,
where the model generates a weighted sum of the input elements that are most
relevant to the task at hand.

Log Probabilities Log probabilities, also known as logprobs, are a measure of
the confidence or probability of a particular output given a set of inputs. In the
context of LLMs, logprobs are used to compute the likelihood of a sequence of
words being generated by the model.

The logprob is calculated by taking the logarithm of the probability
distribution over all possible outputs, given the input. This produces a
numerical value that represents the log-likelihood of the output sequence, with higher
values indicating greater confidence in the predicted output.

Logprobs in LLMs In the context of LLMs, logprobs are used to compute the
uncertainty or confidence in the model's predictions. By analyzing the logprob values
associated with different outputs, it is possible to identify the most likely or
confident predictions made by the model.

The use of logprobs in LLMs has several advantages, including:

Improved decision-making : Logprobs can be used to guide decision-making processes
by providing a quantitative measure of confidence in each option. Uncertainty
quantification : Logprobs can be used to quantify the uncertainty associated
with different predictions or outputs. Efficient resource allocation : Logprobs can
be used to allocate resources more efficiently, as they provide a measure of
confidence that can inform decision-making. In the context of your project,
logprobs will likely play an important role in evaluating the performance of the
LLM-based agent. By analyzing the logprob values associated with different outputs,
you can gain insights into the model's uncertainty and make more informed
decisions about resource allocation.

I hope this helps! Let me know if you have any further questions or need additional
clarification on any of these points.
\section{Log probs Based Uncertainty in LLMs with Conformal Prediction}

Conformal Prediction (CP) is a machine learning framework that provides a statistical
approach to quantify uncertainty in predictive models, including those employed
by Large Language Models (LLMs). In this section, we will delve into the world
of CP and its application to LLMs, highlighting how it has been utilized to compute
log-probability based uncertainty.

Conformal Prediction is rooted in the concept of exchangeability, which states
that a set of data points can be treated as a sample from an infinite population
without any loss of generality. This assumption enables CP to provide prediction
intervals for any point predictor, whether statistical, machine learning, or
deep learning-based models. The core idea behind CP is to compute nonconformity scores
on previously labeled data and use these scores to create prediction sets on new,
unlabelled test data points.

In the context of LLMs, CP has been employed to estimate uncertainty in predictions
made by these complex models. One of the primary challenges in evaluating the
uncertainty of LLMs is the difficulty in directly computing the output
distribution over the entire vocabulary space. Instead, researchers have focused
on developing methods that can effectively quantify uncertainty through
intermediate representations or metrics.

One such approach involves using the expected log-probability (ELP) as a
surrogate for the true probability distribution. The ELP is calculated by taking
the expected value of the log-probability of the predicted output given the input
data. This method has been successfully applied to various LLM architectures, including
transformer-based models, to estimate uncertainty.

A notable example of this approach is the work by Kuhn et al. [38], who proposed
a CP framework for quantifying uncertainty in deep learning models. Their work demonstrates
how CP can be adapted to calculate nonconformity scores on labeled data and use these
scores to construct prediction intervals for unlabelled test data points.

Another relevant contribution is from the work of Vishwakarma et al. [22], who proposed
a conformal prediction approach for uncertainty estimation in LLMs. Their
framework leverages the concept of expected log-probability (ELP) as a proxy for
true probability distribution and demonstrates its effectiveness in quantifying
uncertainty in transformer-based models.

In conclusion, Conformal Prediction provides a robust framework for quantifying uncertainty
in predictive models, including those employed by Large Language Models. By leveraging
intermediate representations or metrics, researchers have developed methods to
effectively estimate log-probability based uncertainty in LLMs. As the field continues
to evolve, it is likely that CP will play an increasingly important role in our
understanding and development of these complex models.
\section{Using Logprobs thanks to OpenAI API}
Log probabilities, also known as logprobs, are a crucial feature of the OpenAI Chat
Completions API that provides valuable insights into the model's output. In this
explanation, we'll delve into how logprobs work with OpenAI APIs and their significance
in understanding the model's behavior.

What are Log Probabilities?

Log probabilities, denoted as logp, represent the logarithm of the probability of
a token occurring at a specific position in the sequence given the context. In
simpler terms, it measures the likelihood of a particular token being chosen by the
model based on the preceding tokens. The log scale is used to avoid dealing with
extremely large probabilities, which can be unwieldy.

How Do Log Probabilities Work?

When you make a request to the OpenAI Chat Completions API, you specify the
context and prompt for the response. The model generates a sequence of tokens that
attempt to fulfill the prompt based on the input context. As the model processes
the input, it assigns probabilities to each possible token, representing the
likelihood of each option being chosen.

Log Probabilities in OpenAI APIs

When logprobs is enabled (set to True), the API returns not only the generated
sequence of tokens but also the corresponding log probabilities for each token. This
means that along with each output token, you receive a logarithmic measure of
its probability, which can be used to evaluate the model's performance.

Understanding Log Probabilities

Log probabilities provide several insights into the model's behavior:

Token likelihood : The higher the log probability value, the more likely the corresponding
token is to appear in the sequence. Contextual understanding : By examining the
log probabilities of each output token, you can gauge how well the model understands
the context and has grasped the nuances of the prompt. Confidence levels : Log
probabilities help you assess the confidence level of the model's prediction. A high
log probability value indicates higher confidence in the chosen token. Using Top
Log Probabilities

The top\_logprobs parameter allows you to specify a limit on the number of most likely
tokens to return at each position (ranging from 0 to 5). When enabled, this feature
provides additional insights into the model's behavior by highlighting the top-performing
tokens with their corresponding log probabilities. This is particularly useful
for evaluating model performance and identifying patterns in the output.

\section{Other way for LLM uncertainty}
The current research on building confidence in LLM (Large Language Model)
outputs is a rapidly evolving field. Log probablistic-based methods are indeed one
of the approaches being explored to improve the trustworthiness and reliability of
LLM-generated content.

In this context, "log probablistic-based methods" refer to techniques that
leverage log probabilities to quantify uncertainty in the model's output. These methods
aim to provide a more nuanced understanding of the confidence or certainty associated
with each prediction made by the LLM.

Some of the key research directions and findings related to log probablistic-based
methods for building confidence in LLM outputs include:

Bayesian Neural Networks (BNNs) : BNNs are a type of neural network that incorporates
Bayesian inference techniques, including log probability calculations, into the model
architecture. This allows BNNs to explicitly represent uncertainty and provide
probabilistic predictions. Uncertainty Estimation using Log Probabilities : Researchers
have proposed various methods for estimating uncertainty in LLM outputs, such as
calculating the log probability of each class or label. These methods can be used
to identify regions of high uncertainty in the output and guide the user's attention
accordingly. Ensemble Methods with Log Probabilistic Outputs : Ensemble methods
combine multiple models' predictions to improve overall performance. Researchers
have explored using log probablistic-based methods to generate ensemble outputs,
which can provide a more robust estimate of confidence in the final prediction.
While these research directions show promise, it's essential to note that building
confidence in LLM outputs is an ongoing challenge. The field is rapidly evolving,
and new techniques are being developed to address this issue.

One notable example of a log probablistic-based method specifically designed for
LLMs is Logistic Monte Carlo Dropout , which uses log probabilities to calculate
uncertainty and guide the model's attention during inference.

Unfortunately, I couldn't find any specific studies or papers that cite all the
log probablistic-based methods currently being researched in this field. However,
I hope this overview provides a general sense of the current state of research
on building confidence in LLM outputs using log probablistic-based methods.