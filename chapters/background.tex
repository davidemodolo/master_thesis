\chapter{Background}

What is needed to understand the project.

\section{Software Agents / Autonomous Agents}

In recent years, the field of Artificial Intelligence (AI) has witnessed a
paradigmatic shift with the emergence of autonomous agents. These intelligent systems
have the ability to perceive their environment, reason about it, and take actions
to achieve their goals without the need for explicit human supervision or intervention.
But in the last year, the focus has shifted towards the integration on top of Large
Language Models (LLMs) in these agents to complete more complex reasoning tasks.
[ADD EXAMPLE OF OpenAI o1 and langchain]

The development of autonomous agents has far-reaching implications for various fields,
including robotics, natural language processing, and game playing.

[Summarize the definition from the slides of the course.]

[talk about the difference between AA and Genetic algorithms]

\section{BDI Architecture}
The Belief-Desire-Intention (BDI) architecture is a popular framework for developing
autonomous agents that can reason about their own mental states and take appropriate
actions in complex environments.

\paragraph{Belief}
The first component of the BDI architecture is Belief (B). This represents the agent's
mental state, which includes all the information that the agent knows or
believes to be true about its environment. In other words, belief is the agent's
internal representation of reality.

For example, if we have an autonomous robot that needs to navigate through a room,
its beliefs might include:
\begin{itemize}[noitemsep]
  \item The location of the robot (e.g., "I am in the living room")

  \item The presence of obstacles (e.g., "There is a chair in front of me")

  \item The goal of the task (e.g., "My goal is to reach the kitchen")
\end{itemize}
The Belief component is responsible for managing and updating the agent's mental
state based on new information or observations. This involves reasoning about
the meaning of the new information, combining it with existing beliefs, and
revising the belief set accordingly.

\paragraph{Desire}
The second component of the BDI architecture is Desire (D). This represents the
agent's goals, preferences, and motivations. Desires are high-level representations
of what the agent wants to achieve or attain in a given situation. Unlike
beliefs, desires are not necessarily based on factual information but rather on the
agent's subjective preferences and values.

For example, if we have an autonomous robot that needs to clean up a room, its desires
might include:

\begin{itemize}[noitemsep]
  \item Cleaning up the room quickly (e.g., "I want to clean up the room as fast
    as possible")

  \item Cleaning up the room completely (e.g., "I want to clean up every single piece
    of trash")

  \item Avoiding certain areas or objects (e.g., "I don't want to go near the breakable
    vase")
\end{itemize}
The Desire component is responsible for evaluating the agent's goals and preferences
in light of its current situation. This involves reasoning about what actions
are likely to achieve the desired outcome, taking into account any constraints or
limitations.

\paragraph{Intention}
The third component of the BDI architecture is Intention (I). This represents the
agent's commitment to take a specific course of action towards achieving the chosen
Desire (goal). An intention is a plan of action that is explicitly stated and
committed to by the agent.

\section{Large Language Models}
Large Language Models (LLMs) are a type of artificial intelligence model that is
specifically designed to process and understand human language. These models use
complex algorithms and large datasets to learn the patterns and structures of language,
allowing them to generate coherent and contextually relevant text.

\section{Log probs Based Uncertainty in LLMs with Conformal Prediction}

\section{Other way for LLM uncertainty}

\section{PDDL}

\section{Docker}