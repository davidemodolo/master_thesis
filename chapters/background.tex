\chapter{Background}
\label{cha:background}

In this thesis, we will analyze in detail the behavior of an LLM as an agent within
a controlled environment.

Before presenting all the work carried out in detail, this chapter aims to
provide a comprehensive explanation of all the theoretical foundations necessary
to understand the steps presented in the following chapters. Starting from a brief
introduction of Artificial Intelligence just to define the boundaries in which we
are working, we will move to the core concepts. In particular, we want to
highlight what an LLM is and how it works, with a special focus on the Attention
mechanism and how the uncertainty of an LLM can be calculated. This will serve
as a basis for correctly interpreting the results analyzed in Section
\ref{cha:results_discussion}.

There will also be a broader discussion on agents in a strict sense and "LLM agents"
to better show the difference between our implementation and what is currently
being discussed over the media.

To better define the context of this thesis, we will examine the main alternative
approaches to solving a logistical problem currently studied in the literature.

\section{Artificial Intelligence}
\label{sec:artificial_intelligence}

Right now in the media, AI is being used as a synonym for Large Language Models.
However, AI is a broader concept that includes many techniques and methodologies.

\#\# Summary of different kind of AI ending with Generative Models

\section{Large Language Models - LLMs}
\label{sec:large_language_models_llms}

\#\# LLMs are generative models released with the paper "Attention is All You
Need"

\subsection{LLMs' Uncertainty}
\label{sub:llms_uncertainty}

Understanding the uncertainty of an LLM is crucial to correctly interpret the text
it generates. If we ask for a yes/no question, it would make a different impact on
us reading "Yes" or "Yes - Uncertainty 49\%". Moreover, this would let us get
some kind of explainability behind these complex and opaque systems.

\#\# cite Hallucinations

\section{Agents}
\label{sec:agents}
\subsection{BDI Architecture}
\label{sub:bdi_architecture}
\section{State of the Art}
\label{sec:state_of_the_art}