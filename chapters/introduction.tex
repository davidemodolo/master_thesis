\chapter{Introduction}
\label{cha:introduction}

This thesis explores the capabilities of Large Language Models (LLMs) in the
context of a logistics problem. Artificial intelligence has made significant strides
in generative systems, particularly with the advent of LLMs, which are capable of
producing coherent and contextually relevant text based on input prompts.
However, their ability to autonomously plan and achieve goals without additional
external structures remains a topic of investigation. The primary aim of this
work is to assess whether LLMs can be effectively utilized as agents in dynamic
environments without leveraging predefined frameworks or knowledge bases.

To achieve this, we first conducted an analysis of existing methodologies in the
literature. Traditional approaches such as PDDL and Reinforcement Learning
provide structured and systematic ways to tackle planning problems. PDDL offers explainability
and efficiency in constrained environments but lacks adaptability, making it
impractical for real-time applications. On the other hand, Reinforcement
Learning is highly adaptable and effective in changing environments but suffers
from issues such as convergence to local optima and lack of explainability.

Recent research has also explored planning capabilities of LLMs, with several studies
investigating their reasoning abilities as well as planning skills.

The specific problem addressed in this thesis involves evaluating the generative
ability of ``raw" LLMs, devoid of external structures, to solve logistics
problems in dynamic settings. [TODO MAKE THIS LONGER]

We also analyzed some ways to evaluate the LLM uncertainty and used the log-probability
based one to evaluate the uncertainty of an LLM in choosing the action. The approach
required to generate a token that referred to a possible action and working on a
bias for the logits to then compute the probability of correct.

[EXPECTED RESULTS]

The thesis is structured as follows:
\begin{itemize}
  \item Chapter \ref{cha:background} provides an overview of the background and related
    work.

  \item Chapter \ref{cha:experiment_setting} describes the experimental setting and
    the environment used.

  \item Chapter \ref{cha:agent_development} details the development of the agent.

  \item Chapter \ref{cha:data_collection} explains the data collection process.

  \item Chapter \ref{cha:results_discussion} presents the results and discusses the
    findings.

  \item Chapter \ref{cha:future_works} outlines potential future works.

  \item Chapter \ref{cha:conclusions} concludes the thesis.
\end{itemize}