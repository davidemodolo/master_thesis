\chapter{Introduction}
\label{cha:examples}

In recent years, there has been a surge of interest in developing Artificial Intelligence
(AI) systems that can interact with complex environments, such as the web, to achieve
specific goals. One of the key challenges in this area is designing effective
Autonomous Agents (AAs) that can navigate and make decisions in uncertain or
dynamic environments.

An Autonomous Agent is a software program that can perceive its environment, reason
about its actions, and act on that reasoning without human intervention. In the context
of web-based platforms like Deliveroo.js, AAs can interact with users, restaurants,
and logistics services to facilitate food delivery. The goal of this project is to
investigate the potential benefits of integrating Large Language Models (LLMs) into
these agents to improve their decision-making capabilities.

This thesis focuses on exploring different locations for incorporating LLMs within
the process of a Delivery Agent in Deliveroo.js, with the aim of identifying the
optimal placement that maximizes performance. To achieve this goal, we will
implement and compare various agent architectures based on different AI planning
strategies, including Uncertainty Alignment for Large Language Model Planners.
The key idea is to produce the next action weighted by log probabilities, as proposed
in the mentioned paper.

The main contributions of this research are twofold. Firstly, we aim to provide a
systematic evaluation of the performance of LLM-based agents in Deliveroo.js compared
to traditional A*-based BDI (Belief-Desire-Intention) agents. Secondly, by
identifying the most effective placement for an LLM within the agent's process, we
hope to shed light on how to better utilize these powerful models in real-world applications.

The results of this research will contribute to a deeper understanding of the potential
benefits and limitations of integrating LLMs into Autonomous Agents in web-based
environments. By exploring the feasibility of this approach, we can pave the way
for future research into more sophisticated AI systems that can interact with complex
online platforms.

This thesis is organized as follows.

% - Chapter~\ref{cha:examples} provides an
- Chapter 1 provides an

- Chapter 2 provides an

- Chapter 3 provides an

- Chapter 4 provides an