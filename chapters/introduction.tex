\chapter{Introduction}
\label{cha:introduction}

This thesis explores the capabilities of Large Language Models (LLMs) in the
context of a logistics problem. Artificial intelligence has made significant strides
in generative systems, particularly with the advent of LLMs, which are capable of
producing coherent and contextually relevant text based on input prompts.
However, their ability to autonomously plan and achieve goals without additional
external structures remains a topic of investigation. The primary aim of this
work is to assess whether LLMs can be effectively utilized as agents in dynamic
environments without leveraging predefined frameworks or knowledge bases.

To achieve this, a thorough analysis of existing methodologies is necessary.
Traditional approaches such as PDDL and Reinforcement Learning provide
structured and systematic ways to tackle planning problems. PDDL offers explainability
and efficiency in constrained environments but lacks adaptability, making it
impractical for real-time applications. On the other hand, Reinforcement
Learning is highly adaptable and effective in changing environments but suffers
from issues such as convergence to local minima and lack of explainability. Recent
research has also explored planning capabilities of LLMs, with several studies
investigating their reasoning abilities and limitations. The specific problem addressed
in this thesis involves evaluating the capacity of an LLM, devoid of external
structures, to solve logistics problems in dynamic settings.

[TODO]