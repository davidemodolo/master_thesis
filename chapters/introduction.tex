\chapter{Introduction}
\label{cha:examples}

In recent years, there has been a surge of interest in developing Artificial Intelligence
(AI) systems that can interact with complex environments, such as the web, to achieve
specific goals. One of the key challenges in this area is designing effective
Autonomous Agents (AAs) that can navigate and make decisions in uncertain or
dynamic environments.

An Autonomous Agent is a software program that can perceive its environment, reason
about its actions, and act on that reasoning on its own. In case of Deliveroo.js,
the environment is a map (represented by a matrix) where the agent can move and
deliver randomly-spawned parcels. There already exist agents that can either plan
and achieve that goal correctly in an old-fashioned way (ADD CIT TO MY PROJECT) or
with the help of Large Language Models (LLMs) (ADD CIT TO OTHER STUDENT PROJECT).

But what if the AA only knows what actions is it capable of performing, but the
environment is not known? The (hypothetical) sensors just return some information
but we don't know how to handle them. And what if we feed such information to an
LLM without much context? Could it be able to achieve any goal? With this
project, I aim to investigate the potential benefits of integrating Large Language
Models (LLMs) into these agents to see if, without any context or knowledge of a
fast-changing environment, they are still able to achieve a goal and what how it
compares to what an "optimal" solution would be.

This thesis focuses on exploring different steps for incorporating LLMs within the
process of a Delivery Agent in Deliveroo.js, to identify the how much extra
information is needed to achieve the goal other than the sensors raw output. Moreover,
I will test an uncertainty-based approach that I developed during my internship
to see if it can help us understand if it's possible to achieve the goal with
less information.

The main contributions of this project are twofold. Firstly, I aim to provide a
systematic evaluation of the performance of LLM-based agents in Deliveroo.js
compared to the "optimal solution" we would get already knowing everything (DFS).
Secondly, by identifying how much freedom we can give to the LLMs and what
information is forcibly needed to achieve the goal.

The results of this thesis will contribute to a deeper understanding of the
potential benefits and limitations of integrating LLMs into Autonomous Agents. Everything
will be open source and easily replicable.

This thesis is organized as follows.

% - Chapter~\ref{cha:examples} provides an
- Chapter 1 provides an

- Chapter 2 provides an

- Chapter 3 provides an

- Chapter 4 provides an