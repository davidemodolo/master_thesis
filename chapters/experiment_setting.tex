\chapter{Experiment Setting}
\label{cha:experiment_setting}

\textbf{ONGOING}

In this chapter, we provide a comprehensive description of the experimental framework
used to evaluate the performance of our agent. We begin by formally defining the
problem, ensuring a clear understanding of the specific aspects of this field that
our work aims to explore.

Next, we offer a detailed explanation of the environment used to simulate the
delivery platform. This section will cover the structure, features, and functionalities
of the web-based platform.

Finally, we discuss the selection of various large language models (LLMs) used
in our experiments, including both tested and untested models. We will provide
insights into the rationale behind our choices.

\section{Problem Definition}
\label{sec:problem_definition}

As widely explained in Section \ref{sub:planning_in_llm}, the recent
advancements and power of LLMs made us wanting to analyze further their ability
to understand a planning/logistic problem and solve it on their own. In this work,
we are interested in understanding strengths and weaknesses of ``raw'' LLMs, without
any additional planning framework (nor path finding nor reasoning) on top of them.
Moreover, we would like to emphasize again the fact that this is a generative approach,
and we want to test the ability of the model on its own.

Our question is: is an agent, that is using an LLM as ``backend'', able to solve
a logistic problem in a unexpected environment? What are the limitations and the
strengths of this approach?

To simulate the unexpected environment, we decided to use a web-based platform that
works with API calls. We decided to do not parse the JSON of the map received, so
that the agent has to understand the map by itself. Moreover, if the API changes,
the agent keeps working since it is not prepared to anything specific from the
server.

Our tests will be explained in detail later in Section X, but in general we want
to test:
\begin{itemize}
  \item The ability of an agent to pickup a parcel in a specific tile;

  \item The ability of an agent to deliver a parcel in a tile it has to derive from
    the raw map.
\end{itemize}

\section{Environment - Deliveroo.js}
\label{sec:environment_deliveroo_js}

Deliveroo.js it's an Educational Game, developed by Marco Robol for the course on
Autonomous Software Agents (ASA) by Prof. Paolo Giorgini built using Treejs\footnote{\url{https://threejs.org/}}.
The code for the server is open and che be accessed on GitHub \footnote{\url{https://github.com/unitn-ASA/Deliveroo.js}}
as well as some example of agents (with different level of complexity)\footnote{\url{https://github.com/unitn-ASA/DeliverooAgent.js}}.

The game can be played even by humans, but it is designed to serve as an
environment for any kind of agent to interact with it using APIs (TODO ADD SOCKET
EXPLANATION).

Even if the game can be played without the graphical interface, it can be useful
to visualize what the current state is at any given moment.

[ADD FIGURE] \label{fig:deliveroo_js}

As we can see from the Figure \ref{fig:deliveroo_js}, the game is a grid of $N \times
M$ tiles where the agent can move. Right now, the $(0, 0)$ cell is in the bottom
left corner.

There are three possible types of cells:
\begin{itemize}
  \item \textcolor{primary}{\textbf{green}}: the agent can move on it. They can
    contain multiple parcels but only one agent at a time;

  \item \textcolor{red}{\textbf{red}}: the agent can move on it and deliver any number
    of parcel it has;

  \item \textcolor{black}{\textbf{black}}: the agent can't move on it and they can't
    contain any parcel (we will not use them in our tests).
\end{itemize}

The functioning is very straight forward:

\begin{itemize}
  \item \textbf{Agents}: there can be any number of agents that can cooperate or
    compete. Each agent has a score that is increased by delivering parcels. They
    are represented as cones with their name on it on the map (`LLM Agent' in
    Figure \ref{fig:deliveroo_js} is ours).

  \item \textbf{Parcels}: they are represented as small squares with a number on
    it. The number is the reward the agent will get by delivering it. They spawn
    in random cells and they can be picked up by the agent. If they are not
    delivered in a certain amount of time, they may disappear.
\end{itemize}

[ADD LISTING SERVER] \label{lst:deliveroo_server_config}

The behavior of the parcels is specified in the server config file (example of
one of ours in Listing \ref{lst:deliveroo_server_config}). Based on the server
setting file, a maximum number of parcels can spawn at the same time (each one
delayed from the previous one by a specific value) with a random reward value.
Then, also based on the configuration file, its reward can decrease over time or
remain constant.

The agent can interact with the environment using the following actions:
\begin{itemize}
  \item \textbf{up}: move up;

  \item \textbf{down}: move down;

  \item \textbf{left}: move left;

  \item \textbf{right}: move right;

  \item \textbf{pickup}: the agent can pickup a parcel in the cell it is in;

  \item \textbf{deliver}: the agent can deliver a parcel in the cell it is in:
    if it is on a delivery zone the parcel will disappear and the reward will be
    added to the player's score, otherwise it will just be dropped on the cell.
\end{itemize}

The server also sends each event to the agent, in specific these are the events we
used in our tests:
\begin{itemize}
  \item \texttt{onMap} (width, height, tiles)

  \item \texttt{onYou} (id, name, x, y, score )

  \item \texttt{onParcelsSensing} async (perceived\_parcels)
\end{itemize}

\section{LLM Models}
\label{sec:llm_models}

\subsection{Open Source}

\subsection{OpenAI}