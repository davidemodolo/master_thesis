\chapter{Conclusions}
\label{cha:conclusions}

In this work, we have explored the challenges and opportunities associated with navigating
grid-based environments using Large Language Models to solve logistics tasks.
Through a series of experiments, we examined both stateless and stateful agent configurations,
highlighting the strengths and limitations of each approach, the former relying only
on a snapshot of the current environment state, while the latter maintains a
record of past interactions.

Our analysis revealed that stateless agent, although capable of reaching the goal,
struggles with uncertainty when inferring spatial relationships solely from the
prompt as the map size grows. This approach has been the foundation of this thesis,
since it allowed us to map the uncertainty over the entire environment, highlighting
limitations and biases linked to the environment description.

In contrast, the stateful agent, which maintains a record of past actions and environments,
shows an improved ability to handle complex navigation tasks by leveraging
historical context. This feature allows it to better understand and adapt to changes
in the environment, recover from suboptimal paths and make more informed
decisions, particularly in dense or larger grid scenarios.

Our findings underscore the importance of leveraging historical context when dealing
with dynamic and complex environments. While stateless approaches offer
simplicity, they may fall short in scenarios where resolving subtle spatial cues
is critical. The stateful framework, despite imposing additional constraints due
to token limitations, demonstrates significant consistency in accuracy and
adaptability.

We were able to visualize some intrinsic biases in how the model perceives the
map, identifying recurrent behaviors in common map zones that exhibit consistently
higher uncertainty. Our experiments demonstrated that more powerful models tend
to exhibit less uncertainty. Additionally, we confirmed that an LLM pays more
attention to the beginning and end of a prompt, providing insights into optimal prompt
structuring. Importantly, we showed that these biases were not due to our methodology,
as we adhered to established literature and conducted tests with a simplified
version of the problem, yielding almost identical results.

\vspace{1mm}
Some limitations to this thesis include the possibility that the prompt,
although designed in accordance with the literature, may not be optimal.
Additionally, the behavior of OpenAI's API \texttt{logit bias} parameter is
subject to changes, which could impact consistency. Most of the tests were
conducted with a single model, and the results were then generalized to all models,
which may not fully capture variations. Furthermore, the environment was not
truly dynamic; while it was treated as such by not relying on a specific parser,
the agent was not actually tested in a fast-changing environment. Context size limitations
also proved to be crucial in the stateful agent's performance.
\vspace{1mm}

Overall, we were able to demonstrate the potential of LLMs' generative capabilities
in grid-based navigation tasks, highlighting the importance of context and
memory in achieving optimal performance, while also identifying common weaknesses
among existing models.

There are also several promising directions for future research, such as:

\begin{itemize}
  \item Exploring alternative ways to leverage uncertainty at each step, such as
    using it as a trigger for an expert system to take over or incorporating it
    as a variable in a more complex function that adjusts the output dynamically
    (e.g., if the delta between the two highest probabilities actions is below a
    certain epsilon, it could be treated as \emph{no uncertainty});

  \item Investigating the use of multimodal models to achieve a richer representation
    of the environment, as analyzing a 2D map as a mere sequence may
    significantly reduce performance;

  \item Testing reasoning-based models, which, while requiring more time per step,
    during which the environment may change, could benefit from future faster models
    capable of reasoning more efficiently;

  \item Integrate Retrieval-Augmented Generation techniques to enhance the agent's
    decision-making process by adding precomputed favorable paths that can be used
    to guide the agent towards the goal in specific scenarios;

  \item Conducting a broader comparison across all the latest available models,
    including open-source ones, to establish a dedicated benchmark for evaluating
    their capabilities;

  \item Continuously testing newer models, given that performance improvements
    have been observed across successive versions.
\end{itemize}