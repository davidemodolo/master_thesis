\chapter*{Abstract}
\label{cha:abstract}
\addcontentsline{toc}{chapter}{Abstract}

With the recent advancements in Large Language Models (LLMs), there has been a growing
interest in developing agents capable of understanding and executing complex tasks.
In this work, we explore the use of LLMs for building an agent that can navigate
and complete logistics tasks, primarily focused on picking up and delivering
parcels, within a web-based environment.

Our approach aims to evaluate the raw performance of an LLM without integrating additional
frameworks or specialized optimization techniques, allowing us to assess its
inherent capabilities in problem-solving. We analyze how effectively the agent can
navigate different map layouts and complete assigned objectives, testing its
adaptability across various goal configurations. A key aspect of our evaluation is
the use of LLM uncertainty measures, derived from tokens' log probabilities, to
gain deeper insights into the model's confidence in its decision-making process.
These measures help us understand when the agent is uncertain and how that uncertainty
correlates with performance in different parts of the scenario.

We demonstrate that the agent's performance improves when using newer LLM versions,
reflecting the continuous advancements in these models. However, we also observe
a decline in performance as the map size increases, suggesting that larger environments
pose challenges that the model struggles to overcome. To structure our approach
effectively, we design the prompt based on established literature, ensuring
alignment with best practices in prompt engineering.

Furthermore, we experiment with two distinct agent configurations: a stateless
agent, which makes decisions solely based on the current state of the environment,
and a stateful agent, which retains memory of past interactions. By comparing
these approaches, we highlight the strengths and limitations of each. The
stateless agent benefits from simplicity and avoids memory-related constraints, but
it may struggle in scenarios where the environment description requires too much
attention. Conversely, the stateful agent provides improved continuity in decision-making
but faces challenges related to context length limitations and potential inconsistencies
in stored information.